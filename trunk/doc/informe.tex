%% LyX 1.4.2 created this file.  For more info, see http://www.lyx.org/.
%% Do not edit unless you really know what you are doing.
\documentclass[spanish]{article}
\usepackage[T1]{fontenc}
\usepackage[latin1]{inputenc}

\makeatletter

%%%%%%%%%%%%%%%%%%%%%%%%%%%%%% LyX specific LaTeX commands.
%% Because html converters don't know tabularnewline
\providecommand{\tabularnewline}{\\}

\usepackage{babel}
\deactivatetilden
\makeatother
\begin{document}
\markboth{left head}{Base de datos - 2do Cuatrimestre 2006} \pagestyle{myheadings}


\title{{\Huge Base de datos}{\normalsize }\\
 {\normalsize }{\small 1er. Cuatrimestre 2006}{\normalsize }\\
 {\normalsize }{\small \vspace{3mm}
} {\Large Proyecto: Minibase}{\normalsize }\\
 {\normalsize }{\Large Informe: Dise�o Detallado}{\normalsize }\\
 {\normalsize }}

\maketitle
\begin{center}{\small N�mero de Grupo:} \\
 {\small Nombre del Grupo: ?}\\
 {\small \vspace{5mm}
}  \textbf{\small Integrantes}\\
 \par\end{center}



\begin{center}\begin{tabular}{|l|l|l|}
\hline 
{\small Apellido y Nombre }&
{\small L.U. }&
{\small Mail }\tabularnewline
\hline 
{\small Leandro Groisman  }&
{\small 222/03 }&
{\small gleandro@gmail.com }\tabularnewline
\hline 
{\small Fernando Rodriguez}&
{\small XXX/XX }&
{\small ferrod20@gmail.com}\tabularnewline
\hline 
{\small Guillermo Amaral}&
{\small 522/98 }&
{\small Guillermo.AMARAL@total.com}\tabularnewline
\hline 
{\small Facioni Francisco }&
{\small 004/04 }&
{\small fran6co@fibertel.com.ar }\tabularnewline
\hline
\end{tabular}\par\end{center}



{\small \vspace{5mm}
 }{\small \par}

\begin{center}\begin{tabular}{|l|l|l|}
\hline 
{\small Instancia }&
{\small Corrector }&
{\small Nota }\tabularnewline
\hline 
{\small Entrega }&
&
\tabularnewline
\hline 
{\small Reentrega }&
&
\tabularnewline
\hline
\end{tabular}\par\end{center}



{\small \vspace{10mm}
 }{\small \par}

{\small Comentarios del corrector: }{\small \par}

\thispagestyle{empty}

\newpage{}

\setcounter{page}{1}

\pagenumbering{roman}

\tableofcontents{}\newpage{}

\setcounter{page}{1}

\pagenumbering{arabic}


\section{Introducci�n a Minibase}

\newpage{}


\section{BufferManager-Heap}


\subsection{Descripci�n general}


\subsection{Clases principales y sus protocolos principales}


\subsection{Interacci�n con otros componentes}


\subsection{Ejemplo de uso}

Si se justifica


\subsubsection{Diagramas de secuencia}


\subsubsection{Script de ejemplo}


\subsection{Evaluaci�n del componente (opini�n acerca de la calidad de c�digo,
dise�o, etc.)}

\newpage{}


\section{DiskManager}


\subsection{Descripci�n general}


\subsection{Clases principales y sus protocolos principales}


\subsection{Interacci�n con otros componentes}


\subsection{Ejemplo de uso}

Si se justifica


\subsubsection{Diagramas de secuencia}


\subsubsection{Script de ejemplo}


\subsection{Evaluaci�n del componente (opini�n acerca de la calidad de c�digo,
dise�o, etc.)}

\newpage{}


\section{Catalog}


\subsection{Descripci�n general}


\subsection{DER}


\subsection{Clases principales y sus protocolos principales}


\subsection{Interacci�n con otros componentes}


\subsection{Ejemplo de uso}

Si se justifica


\subsubsection{Diagramas de secuencia}


\subsubsection{Script de ejemplo}


\subsection{Evaluaci�n del componente (opini�n acerca de la calidad de c�digo,
dise�o, etc.)}

\newpage{}


\section{Iterator}

TODO: Revisar los arboles binarios como estan construido y hacer un
analisis mas profundo

TODO: Realizar una descripcion mas profunda de cada clase

TODO: Mencionar las clases de excepciones que estan desperdigadas
por ahi

TODO: Revisar con mas detenimiento las clases IoBuf y OBuf, q parecen
un poco al pedo

TODO: Analizar mas profundamente las clases de eval y projection,
y en la parte de uso explicar como se introduce una condicion


\subsection{Descripci�n general}

La componente iterator es el punto de acceso a las tablas y a sus
registros. Como tal ofrece, tambien, las operaciones basicas sobre
tablas, como join, proyeccion y seleccion. La iterfaz usada, como
dice el nombre, es la de un iterador, el cual se inicializa en base
a otros iteradores o heapfiles y despues se va accediendo elemento
por elemento en un orden definido por el iterador.


\subsection{Clases principales y sus protocolos principales}

Como dice el nombre de la componente, la interfaz que ofrece es la
de un iterador. Esta interfaz es implementada usando una clase abstracta
de la cual heredan todas las clases que ofrecen acceso a un conjunto
de registros.


\subsubsection{FileScan}

FileScan permite iterar sobre los registros de un heapfile que cumplen
una condicion de seleccion. Esta condicion puede ser nula, permitiendo
iterar sobre todos los registros del heapfile. Tambien se puede especificar
los atributos de las tuplas de salidas, permitiendo hacer una proyeccion.

La evaluaci�n de la condicion de seleccion y la proyeccion son proveidas
por las clases PredEval y Projection, respectivamente.


\subsubsection{NestedLoopsJoins}

NestedLoopsJoins permite realizar un join entre un iterador y un heapfile.
El algoritmo utilizado es el m�s simple de los implementados, es un
doble loop donde, en el cuerpo del loop interno, se verfican la condicion
de join. Tambien se puede realizar una proyeccion en la salida.


\subsubsection{SortMerge}

SortMerge realiza un join utilizando el algoritmo de merge sort. Como
NestedLoopsJoins permite realizar una proyeccion en la salida.

Para realizar el sort utiliza varias clases auxiliares, como Sort
(que se encarga de iterar de una manera ordenada un heapfile) y IoBuf
(que permite almacenar en memoria paginas).

Esta implementacion no elimina los registros duplicados.


\subsubsection{Sort}

Sort permite iterar de una manera ordenada la salida de otro iterador.
Utiliza un arbol binario ordenado para establecer el orden de la salida.


\subsection{Interacci�n con otros componentes}

Esta clase es el punto de acceso principal a los registros. En general
con esta clase es con la cual se realiza toda interaccion con la base
de datos (salvo la modificacion de las tablas o de los registros). 

Respecto a la utilizacion de las otras componentes, Iterator se limita
a utilizar heapfile para la interaccion con las tablas.


\subsection{Ejemplo de uso}

TODO: algun ejemplo de uso q muestre claramente los detalles


\subsubsection{Diagramas de secuencia}

No se justifica


\subsubsection{Script de ejemplo}

?


\subsection{Evaluaci�n del componente}

Iterator ofrece una interfaz poco practica para el uso frecuente.
Mucha funcionalidad esta repetida, como la proyeccion, que simplemente
se podria implementar como otro iterador. Tambien resulta incomoda
la construccion de un iterador por la cantidad de estructuras de datos
que hay que generar. Igualmente esto se debe a pobre dise�o de las
clases y su modularizaci�n a favor de un estilo que se encuentra,
por lo general, en programas escritos en C.

\newpage{}


\section{Index-BTree}


\subsection{Descripci�n general}


\subsection{Clases principales y sus protocolos principales}


\subsection{Interacci�n con otros componentes}


\subsection{Ejemplo de uso}

Si se justifica


\subsubsection{Diagramas de secuencia}


\subsubsection{Script de ejemplo}


\subsection{Evaluaci�n del componente (opini�n acerca de la calidad de c�digo,
dise�o, etc.)}

\newpage{}


\section{Optimizer (no existente!)}


\subsection{Descripci�n general}


\subsection{Clases principales y sus protocolos principales}


\subsection{Interacci�n con otros componentes}


\subsection{Ejemplo de uso}

Si se justifica


\subsubsection{Diagramas de secuencia}


\subsubsection{Script de ejemplo}


\subsection{Evaluaci�n del componente (opini�n acerca de la calidad de c�digo,
dise�o, etc.)}

\newpage{}


\section{Tests}


\subsection{Descripci�n}


\subsection{Resultados obtenidos}


\subsection{Ejemplos de uso}




\section{Herramienta de carga de datos}


\subsection{Descripci�n}


\subsection{Ejemplos de uso}


\section{Conclusiones generales}


\section{Apendices}


\section{C�digo fuente}


\section{Referencias/Bibliograf�a}
\end{document}
